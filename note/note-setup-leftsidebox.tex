%===================================
% note-setup-leftsidebox.tex
% huanengchen@foxmail.com 2025-10-24
%===================================
% 参考:https://tex.stackexchange.com/questions/59702/suggest-a-nice-font-family-for-my-basic-latex-template-text-and-math

%===================================
% 页面和间距
%===================================

\usepackage[a4paper, margin=1in]{geometry} % 具体设置参考 geometry 宏包
\setlength{\parindent}{0pt} % 取消首行缩进
\usepackage{parskip} % 形成段落间的间距
\linespread{1.25} % 修改行距

%===================================
% 章节标题格式控制
%===================================

\usepackage{titlesec} % 用于自定义章节标题格式的宏包

% 重新定义section标题:左对齐、加粗、编号与文本间距1em
\titleformat{\section}
  {\Large \bfseries\leftmargin=0pt} % 左对齐(leftmargin=0pt)+ 加粗
  {\thesection} % 标题前缀
  {1em} % 编号与标题文本的间距
  {} % 标题文本格式(无额外设置,继承前面的bfseries)

%===================================
% 编辑体验
%===================================

\usepackage{float} % 优化浮动体
\usepackage[shortlabels,inline]{enumitem} % 优化列表
\usepackage{appendix} % 优化附录

%===================================
% 表格
%===================================

\usepackage{booktabs, multirow, multicol}
\usepackage{tabularx} % 表格自动换行,调整表格宽度 
\usepackage{makecell} % 单元格内换行
\usepackage{threeparttable} % 给表格添加脚注,参考:https://tex.stackexchange.com/questions/6090/clickable-table-footnote
\usepackage{ltablex} % 跨页表格

%===================================
% 参考文献
%===================================

\usepackage[sort&compress]{gbt7714} % 参考文献样式
\bibliographystyle{gbt7714-numerical} % 顺序编码制

%===================================
% 颜色
%===================================

\usepackage[dvipsnames, x11names, table]{xcolor} % 参考:https://tex.stackexchange.com/questions/659036/option-selecting-named-colours-provided-by-the-xcolor-package

%===================================
% 支持插入图片及子图
%===================================

\usepackage{graphicx}
\graphicspath{
    {./figure/}{./figures/}{./image/}{./images/}{./graphic/}{./graphics/}{./picture/}{./pictures/}
} % 用于存放图片的目录,这样引用图片的时候就不需要指定目录
\usepackage{subcaption}

%===================================
% 算法和伪代码
%===================================

\usepackage[linesnumbered, ruled, longend, lined]{algorithm2e} % 参考 algorithm2e 宏包文档
\DontPrintSemicolon % 不打印分号
\setlength{\algomargin}{2em} % 设置算法缩进使得行号在线框内
\renewcommand{\CommentSty}[1]{\normalsize\textit{#1}} % 设置注释的字体样式为意大利斜体,字体大小为 \normalsize

%===================================
% 代码
%===================================

\usepackage{minted} % 参考 minted 宏包文档

% 代码行号样式
\renewcommand{\theFancyVerbLine}{
\sffamily
\textcolor{gray}{
\footnotesize\oldstylenums{
\arabic{FancyVerbLine}}}}

% 行间代码环境
\setminted{
    style=colorful, % 设置代码风格,可选的代码风格参考:https://pygments.org/styles/
    numbers=left, % 显示行号
    numbersep=2pt, % 行号与代码的距离
    mathescape, % 允许在代码注释中使用数学公式
    breaklines, % 允许代码自动断行 
    fontsize=\footnotesize, % 设置代码字体大小
    frame=single, % 设置代码框
    framerule=0.5pt, % 设置代码框线宽
    resetmargins, % 重置代码边距
}

% 行内代码环境
\setmintedinline{
    style=colorful, % 设置代码风格,可选的代码风格参考:https://pygments.org/styles/
    fontsize=\footnotesize, % 设置代码字体大小
    breakanywhereinlinestretch=0.01em, % 允许行内代码在任意位置断行
    breaklines, % 允许行内代码自动断行
}

%===================================
% 超链接
%===================================

\usepackage{hyperref}
\hypersetup{
    bookmarksopen=true, % 启用书签
    colorlinks=true, % 启用颜色
    linkcolor=red, % 内部链接的颜色
    linktoc=all, % 设置目录中的页码和标题都能够跳转
    citecolor=violet, % 引用链接的颜色
    urlcolor=magenta, % 外部链接的颜色
}

% 自定义 autoref 的引用格式
\def\figureautorefname{图} % 将 "Figure" 改为 "图"
\def\tableautorefname{表}  % 将 "Table" 改为 "表"
\def\equationautorefname{公式} % 将 "Equation" 改为 "公式"

%===================================
% 数学公式
%===================================

\usepackage{amsmath, amsthm, amsfonts, amssymb} % 基础数学包
\usepackage{mathtools}
\usepackage{mathrsfs}
\usepackage{bm}
\usepackage{extarrows}
% \usepackage{lmodern} % 避免与newtx系列字体冲突
\usepackage{newtxtext} % newtx文字字体(需与newtxmath配套使用)
\usepackage{newtxmath} % 设置数学字体为newtxmath
\usepackage[noabbrev]{cleveref} % 多公式引用,必须放在 hyperref 宏包的后面,参考:https://tex.stackexchange.com/questions/314217/how-i-can-refer-multiple-equation-in-latex

\allowdisplaybreaks[1] % 多行公式换页,1 为尽量避免换页
\crefname{equation}{}{} % 设置非首字母大写的引用格式
\Crefname{equation}{}{} % 设置首字母大写的引用格式
\crefrangeformat{equation}{(#3#1#4)-(#5#2#6)} % 多公式引用的格式

%===================================
% 边注
%===================================

% 设置边注的字体大小
\let\oldmarginpar\marginpar
\renewcommand{\marginpar}[1]{\oldmarginpar{\footnotesize #1}}

%===================================
% 页脚和页眉
%===================================

\usepackage{fancyhdr} % 参考:https://tex.stackexchange.com/questions/732462/chapter-number-in-the-header-with-chapter/732464?noredirect=1#comment1824660_732464 
\usepackage{lastpage} % 获取总页码
\setlength{\headheight}{13pt} % 设置页眉高度

% 重新定义 \author 和 \date 命令用于页眉
\makeatletter
\let\oldauthor\author
\renewcommand{\author}[1]{\oldauthor{#1}\def\myauthor{#1}}
\let\olddate\date
\renewcommand{\date}[1]{\olddate{#1}\def\mydate{#1}}

% 重定义\maketitle:调整标题/作者/日期 + 与页眉间距
\renewcommand{\maketitle}{
    \vspace{-1cm} % 调整标题与页眉的间距(负数值缩小间距,可按需微调)
    \begin{center} % 标题居中对齐
        {\Huge \bfseries \@title \par} % 标题加粗放大
    \end{center}
    % 刻意删除 \@author 和 \@date 隐藏不必要的信息
    % \begin{center} % 居中对齐
    %     {\@author \par}
    %     {\@date \par}
    % \end{center}
}
\makeatother

% 详细参数参考 fancyhdr 宏包
\pagestyle{fancy} % 设置文档的页面样式为 fancy
\fancyhf{} % 清空原本的页脚页眉样式

% 自定义页眉
\fancyhead[L]{\myauthor} % 左侧显示作者
\fancyhead[R]{\mydate} % 右侧显示日期

\cfoot{\thepage\ / \pageref*{LastPage}} % 自定义页脚,参考:https://tex.stackexchange.com/questions/227/how-can-i-add-page-of-on-my-document


% 重定义首页的 plain 样式,使其与 fancy 样式一致
\fancypagestyle{plain}{
    \fancyhf{} % 清空默认 plain 样式的页眉页脚
    \fancyhead[L]{\myauthor} % 首页页眉左侧显示作者
    \fancyhead[R]{\mydate} % 首页页眉右侧显示日期
    \cfoot{\thepage\ / \pageref*{LastPage}} % 首页页脚保持与正文一致
}

%===================================
% 字体
%===================================

\usepackage[T1]{fontenc} % 改善文档中西欧语言的显示效果
\usepackage{anyfontsize}
\usepackage{libertine} % Linux Libertine 字体系列(衬线字体)

% 中文字体,具体设置参考 ctex 宏包
\setCJKmainfont{LXGWWenKaiScreen.ttf}[ % 设置中文主字体为霞鹜文楷屏幕舒适版
    Path=./fonts/,
    BoldFont=FZHTJW.TTF, % 设置粗体为方正黑体
    ItalicFont=FZKTJW.TTF, % 设置斜体为方正楷体
]
\setCJKsansfont{FZHTJW.TTF}[ % 设置无衬线字体为方正黑体
    Path=./fonts/,
    AutoFakeBold=1.5,  % 生成粗体效果
    ItalicFont=FZFSJW.TTF, % 设置斜体为方正仿宋
]
\setCJKmonofont{LXGWWenKaiMonoScreen.ttf}[ % 设置等宽字体为霞鹜文楷等宽屏幕舒适版
    Path=./fonts/,
    AutoFakeBold=1.5,  % 生成粗体效果
    ItalicFont=FZSSJW.TTF, % 设置斜体为方正书宋
] 

% 英文字体,具体设置参考 fontspec 宏包
\setmonofont{MapleMono-NF-CN}[ % 设置英文等宽字体
    Path=./fonts/, % 指定字体文件所在的目录
    Extension=.ttf, % 字体文件后缀
    UprightFont=*-Regular, % 正常字体
    BoldFont=*-Bold, % 加粗
    ItalicFont=*-Italic, % 斜体
    BoldItalicFont=*-BoldItalic, % 粗斜体
]

%===================================
% 定理盒子
%===================================

% 修改 proof 环境的引导词为 Proof,样式为加粗无斜体
\renewcommand*{\proofname}{\normalfont\bfseries Proof}

% 导入 thmtools 宏包,使用 \declaretheorem 命令来定义各种定理环境(比 \newtheorem 命令更加方便)
\usepackage{thmtools}

% 定义环境使用的 `\declaretheorem` 命令参数包括:
% - `style`: 定理环境样式,amsthm 内置的样式包括
%   - plain(默认):引导词是正体,内容是斜体
%   - definition:引导词和内容都是正体
%   - remark:引导词是斜体,内容是正体
% - `name`:显示在正文中的引导词(不等于环境的名称)
% - `numbered`:是否开启编号
% - `numberwithin`、`sibling`:定义编号规则,例如:
%   - `numberwithin=section`:基于 section 编号
%   - `sibling=theorem`:共享 `theorem` 环境的编号

% 采用 plain 样式,定义 `theorem`/`theorem*`、`proposition`/`proposition*`、`corollary`/`corollary*`、`lemma`/`lemma*`、`claim`/`claim*` 环境

\declaretheorem[style=plain, name=Theorem, numbered=yes, numberwithin=section]{theorem}
\declaretheorem[style=plain, name=Theorem, numbered=no]{theorem*}

\declaretheorem[style=plain, name=Proposition, numbered=yes, sibling=theorem]{proposition}
\declaretheorem[style=plain, name=Proposition, numbered=no]{proposition*}

\declaretheorem[style=plain, name=Corollary, numbered=yes, sibling=theorem]{corollary}
\declaretheorem[style=plain, name=Corollary, numbered=no]{corollary*}

\declaretheorem[style=plain, name=Lemma, numbered=yes, sibling=theorem]{lemma}
\declaretheorem[style=plain, name=Lemma, numbered=no]{lemma*}

\declaretheorem[style=plain, name=Claim, numbered=yes, sibling=theorem]{claim}
\declaretheorem[style=plain, name=Claim, numbered=no]{claim*}

% 采用 definition 样式,定义 `definition`/`definition*`、`example`/`example*`、`problem`/`problem*` 环境

\declaretheorem[style=definition, name=Definition, numbered=yes, numberwithin=section]{definition}
\declaretheorem[style=definition, name=Definition, numbered=no]{definition*}

\declaretheorem[style=definition, name=Example, numbered=yes, numberwithin=section]{example}
\declaretheorem[style=definition, name=Example, numbered=no]{example*}

\declaretheorem[style=definition, name=Problem, numbered=yes, numberwithin=section]{problem}
\declaretheorem[style=definition, name=Problem, numbered=no]{problem*}

% 采用 remark 样式,定义 `remark`/`remark*`、`note`/`note*` 环境

\declaretheorem[style=remark, name=Remark, numbered=yes, numberwithin=section]{remark}
\declaretheorem[style=remark, name=Remark, numbered=no]{remark*}

\declaretheorem[style=remark, name=Note, numbered=yes, numberwithin=section]{note}
\declaretheorem[style=remark, name=Note, numbered=no]{note*}

% 使用 `\declaretheoremstyle` 命令定义新的 solutionstyle 样式,类似 proof 环境,但是引导词变成 Solution
\declaretheoremstyle[headfont=\bfseries, bodyfont=\normalfont, spaceabove=3pt, spacebelow=3pt, qed=\ensuremath{\square}]{solutionstyle}

% 采用新定义的 solutionstyle 样式,定义 `solution`/`solition*` 环境
\declaretheorem[style=solutionstyle, name=Solution, numbered=yes, numberwithin=section]{solution}
\declaretheorem[style=solutionstyle, name=Solution, numbered=no]{solution*}

% 导入 tcolorbox 宏包以使用盒子美化现有的定理环境
\usepackage[most]{tcolorbox}

% tcolorbox 宏包的功能非常复杂,这里只需要使用 `\tcolorboxenvironment` 命令
% 首先封装一个 `\newtcbenvironment` 命令
% 它可以同时为 `#1` 以及 `#1*` 这两个环境加上盒子,公共参数:
% - `#2`:在定义时传入的参数,这里主要是边框颜色和背景色
% - `enhanced`:样式增强
% - `breakable`:允许跨页
% - `boxrule=1pt`:边框宽度为 1pt
%
% 还有不同的参数:
% - `#1` 盒子使用直角边框(`sharp corners`)
% - `#1*` 盒子使用圆角边框(`rounded corners`)
%
% > 对 `\newtcbenvironment` 内部的公共参数部分进行调整,就可以实现所有盒子只保留左侧边框或者四周无边框等不同的效果。

\newcommand{\newtcbenvironment}[2]{
    \tcolorboxenvironment{#1}{#2, enhanced, breakable, sharp corners,leftrule=2pt, rightrule=0pt, toprule=0pt, bottomrule=0pt}
    \tcolorboxenvironment{#1*}{#2, enhanced, breakable, rounded corners,leftrule=2pt, rightrule=0pt, toprule=0pt, bottomrule=0pt}
}

% 下面就是为前面的各种定理环境加上盒子,参数是盒子的边框颜色 `colframe` 和背景色 `colback`
%
% 具体颜色如下表
%
% |            环境名             |   盒子边框颜色    |    盒子背景色    |
% | :---------------------------: | :---------------: | :--------------: |
% |   `theorem`, `proposition`    |    RoyalPurple    |  RoyalPurple!8   |
% | `corollary`, `lemma`, `claim` |     NavyBlue      |    SkyBlue!8     |
% |         `definition`          |    ForestGreen    |  ForestGreen!5   |
% |           `example`           |     RawSienna     |   RawSienna!5    |
% |           `problem`           | WildStrawberry!30 | WildStrawberry!5 |
%
% 说明:
%
% - 这里采用 `xcolor` 宏包所提供的标准颜色,`xx!n`代表将颜色 `xx` 以 `n%` 比例和白色混合得到的浅颜色。
% - 为了避免颜色过多,对语义类似的环境合并采用相同的盒子颜色。

\newtcbenvironment{theorem}{colframe=RoyalPurple, colback=RoyalPurple!8}
\newtcbenvironment{proposition}{colframe=RoyalPurple, colback=RoyalPurple!8}
\newtcbenvironment{corollary}{colframe=NavyBlue, colback=SkyBlue!8}
\newtcbenvironment{lemma}{colframe=NavyBlue, colback=SkyBlue!8}
\newtcbenvironment{claim}{colframe=NavyBlue, colback=SkyBlue!8}

\newtcbenvironment{definition}{colframe=ForestGreen, colback=ForestGreen!5}
\newtcbenvironment{example}{colframe=RawSienna, colback=RawSienna!5}
\newtcbenvironment{problem}{colframe=WildStrawberry!30, colback=WildStrawberry!5}
