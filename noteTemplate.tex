%%%%%%%%%%%%%%%%导入宏包%%%%%%%%%%%%%%%%%%%%


\usepackage{ctex}
\usepackage{syntonly} % 编译LaTeX文件后不生成DVI或者PDF文档,只排查错误,要生成文档可以注释\syntaxonly命令
% \syntaxonly
\usepackage{graphicx} % 插入图片
% \graphicspath{{figures/}{logo/}} % 用于存放图片的目录figures和logo,这样引用图片的时候就不需要指定目录


%%%%%%%%%%%%%%%%%%%%%%%%%%%%%%%%%%%%%%%%%%%%



%%%%%%%%%%%%%%%%设置字体%%%%%%%%%%%%%%%%%%%%


% 中文字体,具体设置参考ctex宏包
\setCJKmainfont{FZSSJW.TTF}[ % 设置中文主字体为方正书宋 
    Path=./fonts/,
    BoldFont=FZHTJW.TTF, % 设置粗体为方正黑体
    ItalicFont=FZKTJW.TTF, % 设置斜体为方正楷体
]
\setCJKsansfont{FZHTJW.TTF}[Path=./fonts/] % 设置无衬线字体为方正黑体
\setCJKmonofont{FZFSJW.TTF}[Path=./fonts/] % 设置等宽字体为方正仿宋

% 英文字体,具体设置参考fontspec宏包
\setmainfont{CaskaydiaCoveNerdFont}[ % 设置英文主字体为CaskaydiaCove Nerd Font
    Path=./fonts/, % 指定字体文件所在的目录
    Extension=.ttf, % 字体文件后缀
    UprightFont=*-regular, % 正常字体
    BoldFont=*-Bold, % 加粗
    ItalicFont=*-Italic, % 斜体
    BoldItalicFont=*-BoldItalic]
\setsansfont{CaskaydiaCoveNerdFont}[ % 设置英文无衬线字体为CaskaydiaCove Nerd Font
    Path=./fonts/, % 指定字体文件所在的目录
    Extension=.ttf, % 字体文件后缀
    UprightFont=*-regular, % 正常字体
    BoldFont=*-Bold, % 加粗
    ItalicFont=*-Italic, % 斜体
    BoldItalicFont=*-BoldItalic]
\setmonofont{CaskaydiaCoveNerdFontMono}[ % 设置英文等宽字体为CaskaydiaCove Nerd Font Mono
    Path=./fonts/, % 指定字体文件所在的目录
    Extension=.ttf, % 字体文件后缀
    UprightFont=*-regular, % 正常字体
    BoldFont=*-Bold, % 加粗
    ItalicFont=*-Italic, % 斜体
    BoldItalicFont=*-BoldItalic]


%%%%%%%%%%%%%%%%设置字体%%%%%%%%%%%%%%%%%%%%


